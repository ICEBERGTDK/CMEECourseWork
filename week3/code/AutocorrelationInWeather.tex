\documentclass[10pt]{article}
\usepackage[utf8]{inputenc}
\usepackage{listings}
\usepackage{xcolor}
\lstset{
    numbers=left,numberstyle=\tiny,
    keywordstyle=\color{blue},commentstyle=\color[cmyk]{1,0,1,0},frame=single,escapeinside=``,extendedchars=false,
    xleftmargin=2em,xrightmargin=2em,aboveskip=1em,tabsize=4,showspaces=false
}

\title{AutocorrelationInWeather}
\author{Dengkui Tang}
\date{October 2020}

\begin{document}

\maketitle

\section{Compute the appropriate correlation coefficient between successive years and store it.}
    \subsection{Load the data from "KeyWestAnnualMeanTemperature.RData"}
    \begin{lstlisting}[language={R}]
    KWAMT = load("../data/KeyWestAnnualMeanTemperature.RData")
    \end{lstlisting}
    \subsection{Read and store data for each column(Year,Temp)}
    \begin{lstlisting}[language={R}]
    year=ats$Year
    temp=ats$Temp
    \end{lstlisting}
    \subsection{Calculate(cor())}
    \begin{lstlisting}[language={R}]
    result=cor(temp1,temp2)
    \end{lstlisting}

\section{Repeat this calculation 10000 times by - randomly permuting the time series, and then recalculating the correlation coefficient for each randomly permuted year sequence and storing it}
    \subsection{Disorganize "Temp" (use the sample function)}
    \begin{lstlisting}[language={R}]
    ts=sample(temp)
    \end{lstlisting}
    \subsection{Calculate(the same as 1.c)}
    \begin{lstlisting}[language={R}]
    ts1=ts[-1]
    ts2=ts[-length(ts)]
    \end{lstlisting}
    \subsection{Loop 10000 times and draw a diagram}
    \begin{lstlisting}[language={R}]
    i=1
    while(i<=10000){
        Tresult[[i]]<-cor(ts1,ts2)
    }
    hist(Tresult)
    \end{lstlisting}

\section{Calculate what fraction of the correlation coefficients from the previous step were greater than that from step1.}
    \subsection{Create a vector}
    \begin{lstlisting}[language={R}]
    Greater<-vector()
    \end{lstlisting}
    \subsection{Use an "if" to judge whether it is greater than "result", and add it into "Greater".}
    \begin{lstlisting}[language={R}]
    if(Tresult[[i]]>result){
        Greater[[j]]<-Tresult[[i]]
        j=j+1
    }
    \end{lstlisting}
    \subsection{judge whether "Greater" is empty or not}
    \begin{lstlisting}[language={R}]
    if(length(Greater)>0){
        print(length(Greater)/10000)
    }else{
        print("No greater value.")
    }
    \end{lstlisting}

\end{document}
